\documentclass{article}

\usepackage{imakeidx}
\usepackage[pdfborder={0 0 0}]{hyperref}
\usepackage{longtable}

\title{bcachefs: Principles of Operation}
\author{Kent Overstreet}

\date{}

\begin{document}

\maketitle
\tableofcontents

\section{Introduction and overview}

Bcachefs is a modern, general purpose, copy on write filesystem descended from
bcache, a block layer cache.

The internal architecture is very different from most existing filesystems where
the inode is central and many data structures hang off of the inode. Instead,
bcachefs is architected more like a filesystem on top of a relational database,
with tables for the different filesystem data types - extents, inodes, dirents,
xattrs, et cetera.

bcachefs supports almost all of the same features as other modern COW
filesystems, such as ZFS and btrfs, but in general with a cleaner, simpler,
higher performance design.

\subsection{Performance overview}

The core of the architecture is a very high performance and very low latency b+
tree, which also is not a conventional b+ tree but more of hybrid, taking
concepts from compacting data structures: btree nodes are very large, log
structured, and compacted (resorted) as necessary in memory. This means our b+
trees are very shallow compared to other filesystems.

What this means for the end user is that since we require very few seeks or disk
reads, filesystem latency is extremely good - especially cache cold filesystem
latency, which does not show up in most benchmarks but has a huge impact on real
world performance, as well as how fast the system "feels" in normal interactive
usage. Latency has been a major focus throughout the codebase - notably, we have
assertions that we never hold b+ tree locks while doing IO, and the btree
transaction layer makes it easily to aggressively drop and retake locks as
needed - one major goal of bcachefs is to be the first general purpose soft
realtime filesystem.

Additionally, unlike other COW btrees, btree updates are journalled. This
greatly improves our write efficiency on random update workloads, as it means
btree writes are only done when we have a large block of updates, or when
required by memory reclaim or journal reclaim.

\subsection{Bucket based allocation}

As mentioned bcachefs is descended from bcache, where the ability to efficiently
invalidate cached data and reuse disk space was a core design requirement. To
make this possible the allocator divides the disk up into buckets, typically
512k to 2M but possibly larger or smaller. Buckets and data pointers have
generation numbers: we can reuse a bucket with cached data in it without finding
and deleting all the data pointers by incrementing the generation number.

In keeping with the copy-on-write theme of avoiding update in place wherever
possible, we never rewrite or overwrite data within a bucket - when we allocate
a bucket, we write to it sequentially and then we don't write to it again until
the bucket has been invalidated and the generation number incremented.

This means we require a copying garbage collector to deal with internal
fragmentation, when patterns of random writes leave us with many buckets that
are partially empty (because the data they contained was overwritten) - copy GC
evacuates buckets that are mostly empty by writing the data they contain to new
buckets. This also means that we need to reserve space on the device for the
copy GC reserve when formatting - typically 8\% or 12\%.

There are some advantages to structuring the allocator this way, besides being
able to support cached data:
\begin{itemize}
	\item By maintaining multiple write points that are writing to different buckets,
		we're able to easily and naturally segregate unrelated IO from different
		processes, which helps greatly with fragmentation.

	\item The fast path of the allocator is essentially a simple bump allocator - the
		disk space allocation is extremely fast

	\item Fragmentation is generally a non issue unless copygc has to kick
		in, and it usually doesn't under typical usage patterns. The
		allocator and copygc are doing essentially the same things as
		the flash translation layer in SSDs, but within the filesystem
		we have much greater visibility into where writes are coming
		from and how to segregate them, as well as which data is
		actually live - performance is generally more predictable than
		with SSDs under similar usage patterns.

	\item The same algorithms will in the future be used for managing SMR
		hard drives directly, avoiding the translation layer in the hard
		drive - doing this work within the filesystem should give much
		better performance and much more predictable latency.
\end{itemize}

\section{Feature overview}

\subsection{IO path options}

Most options that control the IO path can be set at either the filesystem level
or on individual inodes (files and directories). When set on a directory via the
\texttt{bcachefs attr} command, they will be automatically applied recursively.

\subsubsection{Checksumming}

bcachefs supports both metadata and data checksumming - crc32c by default, but
stronger checksums are available as well. Enabling data checksumming incurs some
performance overhead - besides the checksum calculation, writes have to be
bounced for checksum stability (Linux generally cannot guarantee that the buffer
being written is not modified in flight), but reads generally do not have to be
bounced.

Checksum granularity in bcachefs is at the level of individual extents, which
results in smaller metadata but means we have to read entire extents in order to
verify the checksum. By default, checksummed and compressed extents are capped
at 64k. For most applications and usage scenarios this is an ideal trade off, but
small random \texttt{O\_DIRECT} reads will incur significant overhead. In the
future, checksum granularity will be a per-inode option.

\subsubsection{Encryption}

bcachefs supports authenticated (AEAD style) encryption - ChaCha20/Poly1305.
When encryption is enabled, the poly1305 MAC replaces the normal data and
metadata checksums. This style of encryption is superior to typical block layer
or filesystem level encryption (usually AES-XTS), which only operates on blocks
and doesn't have a way to store nonces or MACs. In contrast, we store a nonce
and cryptographic MAC alongside data pointers - meaning we have a chain of trust
up to the superblock (or journal, in the case of unclean shutdowns) and can
definitely tell if metadata has been modified, dropped, or replaced with an
earlier version - replay attacks are not possible.

Encryption can only be specified for the entire filesystem, not per file or
directory - this is because metadata blocks do not belong to a particular file.
All metadata except for the superblock is encrypted.

In the future we'll probably add AES-GCM for platforms that have hardware
acceleration for AES, but in the meantime software implementations of ChaCha20
are also quite fast on most platforms.

\texttt{scrypt} is used for the key derivation function - for converting the
user supplied passphrase to an encryption key.

To format a filesystem with encryption, use
\begin{quote} \begin{verbatim}
bcachefs format --encrypted /dev/sda1
\end{verbatim} \end{quote}

You will be prompted for a passphrase. Then, to use an encrypted filesystem
use the command
\begin{quote} \begin{verbatim}
bcachefs unlock /dev/sda1
\end{verbatim} \end{quote}

You will be prompted for the passphrase and the encryption key will be added to
your in-kernel keyring; mount, fsck and other commands will then work as usual.

The passphrase on an existing encrypted filesystem can be changed with the
\texttt{bcachefs set-passphrase} command. To permanently unlock an encrypted
filesystem, use the \texttt{bcachefs remove-passphrase} command - this can be
useful when dumping filesystem metadata for debugging by the developers.

There is a \texttt{wide\_macs} option which controls the size of the
cryptographic MACs stored on disk. By default, only 80 bits are stored, which
should be sufficient security for most applications. With the
\texttt{wide\_macs} option enabled we store the full 128 bit MAC, at the cost of
making extents 8 bytes bigger.

\subsubsection{Compression}

bcachefs supports gzip, lz4 and zstd compression. As with data checksumming, we
compress entire extents, not individual disk blocks - this gives us better
compression ratios than other filesystems, at the cost of reduced small random
read performance.

Data can also be compressed or recompressed with a different algorithm in the
background by the rebalance thread, if the \texttt{background\_compression}
option is set.

\subsection{Multiple devices}

bcachefs is a multi-device filesystem. Devices need not be the same size: by
default, the allocator will stripe across all available devices but biasing in
favor of the devices with more free space, so that all devices in the filesystem
fill up at the same rate. Devices need not have the same performance
characteristics: we track device IO latency and direct reads to the device that
is currently fastest.

\subsubsection{Replication}

bcachefs supports standard RAID1/10 style redundancy with the
\texttt{data\_replicas} and \texttt{metadata\_replicas} options. Layout is not
fixed as with RAID10: a given extent can be replicated across any set of
devices; the \texttt{bcachefs fs usage} command shows how data is replicated
within a filesystem.

\subsubsection{Erasure coding}

bcachefs also supports Reed-Solomon erasure coding - the same algorithm used by
most RAID5/6 implementations) When enabled with the \texttt{ec} option, the
desired redundancy is taken from the \texttt{data\_replicas} option - erasure
coding of metadata is not supported.

Erasure coding works significantly differently from both conventional RAID
implementations and other filesystems with similar features. In conventional
RAID, the "write hole" is a significant problem - doing a small write within a
stripe requires the P and Q (recovery) blocks to be updated as well, and since
those writes cannot be done atomically there is a window where the P and Q
blocks are inconsistent - meaning that if the system crashes and recovers with a
drive missing, reconstruct reads for unrelated data within that stripe will be
corrupted.

ZFS avoids this by fragmenting individual writes so that every write becomes a
new stripe - this works, but the fragmentation has a negative effect on
performance: metadata becomes bigger, and both read and write requests are
excessively fragmented. Btrfs's erasure coding implementation is more
conventional, and still subject to the write hole problem.

bcachefs's erasure coding takes advantage of our copy on write nature - since
updating stripes in place is a problem, we simply don't do that. And since
excessively small stripes is a problem for fragmentation, we don't erasure code
individual extents, we erasure code entire buckets - taking advantage of bucket
based allocation and copying garbage collection.

When erasure coding is enabled, writes are initially replicated, but one of the
replicas is allocated from a bucket that is queued up to be part of a new
stripe. When we finish filling up the new stripe, we write out the P and Q
buckets and then drop the extra replicas for all the data within that stripe -
the effect is similar to full data journalling, and it means that after erasure
coding is done the layout of our data on disk is ideal.

Since disks have write caches that are only flushed when we issue a cache flush
command - which we only do on journal commit - if we can tweak the allocator so
that the buckets used for the extra replicas are reused (and then overwritten
again) immediately, this full data journalling should have negligible overhead -
this optimization is not implemented yet, however.

\subsubsection{Device labels and targets}

By default, writes are striped across all devices in a filesystem, but they may
be directed to a specific device or set of devices with the various target
options. The allocator only prefers to allocate from devices matching the
specified target; if those devices are full, it will fall back to allocating
from any device in the filesystem.

Target options may refer to a device directly, e.g.
\texttt{foreground\_target=/dev/sda1}, or they may refer to a device label. A
device label is a path delimited by periods - e.g. ssd.ssd1 (and labels need not
be unique). This gives us ways of referring to multiple devices in target
options: If we specify ssd in a target option, that will refer to all devices
with the label ssd or labels that start with ssd. (e.g. ssd.ssd1, ssd.ssd2).

Four target options exist. These options all may be set at the filesystem level
(at format time, at mount time, or at runtime via sysfs), or on a particular
file or directory:

\begin{description}
	\item \texttt{foreground\_target}: normal foreground data writes, and
		metadata if \\ \texttt{metadata\_target} is not set
	\item \texttt{metadata\_target}: btree writes
	\item \texttt{background\_target}: If set, user data (not metadata) will
		be moved to this target in the background
	\item\texttt{promote\_target}: If set, a cached copy will be added to
		this target on read, if none exists
\end{description}

\subsubsection{Caching}

When an extent has multiple copies on different devices, some of those copies
may be marked as cached. Buckets containing only cached data are discarded as
needed by the allocator in LRU order.

When data is moved from one device to another according to the \\
\texttt{background\_target} option, the original copy is left in place but
marked as cached. With the \texttt{promote\_target} option, the original copy is
left unchanged and the new copy on the \texttt{promote\_target} device is marked
as cached.

To do writeback caching, set \texttt{foreground\_target} and
\texttt{promote\_target} to the cache device, and \texttt{background\_target} to
the backing device. To do writearound caching, set \texttt{foreground\_target}
to the backing device and \texttt{promote\_target} to the cache device.

\subsubsection{Durability}

Some devices may be considered to be more reliable than others. For example, we
might have a filesystem composed of a hardware RAID array and several NVME flash
devices, to be used as cache. We can set replicas=2 so that losing any of the
NVME flash devices will not cause us to lose data, and then additionally we can
set durability=2 for the hardware RAID device to tell bcachefs that we don't
need extra replicas for data on that device - data on that device will count as
two replicas, not just one.

The durability option can also be used for writethrough caching: by setting
durability=0 for a device, it can be used as a cache and only as a cache -
bcachefs won't consider copies on that device to count towards the number of
replicas we're supposed to keep.

\subsection{Reflink}

bcachefs supports reflink, similarly to other filesystems with the same feature.
cp --reflink will create a copy that shares the underlying storage. Reading from
that file will become slightly slower - the extent pointing to that data is
moved to the reflink btree (with a refcount added) and in the extents btree we
leave a key that points to the indirect extent in the reflink btree, meaning
that we now have to do two btree lookups to read from that data instead of just
one.

\subsection{Inline data extents}

bcachefs supports inline data extents, controlled by the \texttt{inline\_data}
option (on by default). When the end of a file is being written and is smaller
than half of the filesystem blocksize, it will be written as an inline data
extent. Inline data extents can also be reflinked (moved to the reflink btree
with a refcount added): as a todo item we also intend to support compressed
inline data extents.

\subsection{Subvolumes and snapshots}

bcachefs supports subvolumes and snapshots with a similar userspace interface as
btrfs. A new subvolume may be created empty, or it may be created as a snapshot
of another subvolume. Snapshots are writeable and may be snapshotted again,
creating a tree of snapshots.

Snapshots are very cheap to create: they're not based on cloning of COW btrees
as with btrfs, but instead are based on versioning of individual keys in the
btrees. Many thousands or millions of snapshots can be created, with the only
limitation being disk space.

The following subcommands exist for managing subvolumes and snapshots:
\begin{itemize}
	\item \texttt{bcachefs subvolume create}: Create a new, empty subvolume
	\item \texttt{bcachefs subvolume destroy}: Delete an existing subvolume
		or snapshot
	\item \texttt{bcachefs subvolume snapshot}: Create a snapshot of an
		existing subvolume
\end{itemize}

A subvolume can also be deleting with a normal rmdir after deleting all the
contents, as with \texttt{rm -rf}. Still to be implemented: read-only snapshots,
recursive snapshot creation, and a method for recursively listing subvolumes.

\subsection{Quotas}

bcachefs supports conventional user/group/project quotas. Quotas do not
currently apply to snapshot subvolumes, because if a file changes ownership in
the snapshot it would be ambiguous as to what quota data within that file
should be charged to.

When a directory has a project ID set it is inherited automatically by
descendants on creation and rename. When renaming a directory would cause the
project ID to change we return -EXDEV so that the move is done file by file, so
that the project ID is propagated correctly to descendants - thus, project
quotas can be used as subdirectory quotas.

\section{Management}

\subsection{Formatting}

To format a new bcachefs filesystem use the subcommand \texttt{bcachefs
format}, or \texttt{mkfs.bcachefs}. All persistent filesystem-wide options can
be specified at format time. For an example of a multi device filesystem with
compression, encryption, replication and writeback caching:
\begin{quote} \begin{verbatim}
bcachefs format --compression=lz4               \
                --encrypted                     \
                --replicas=2                    \
                --label=ssd.ssd1 /dev/sda       \
                --label=ssd.ssd2 /dev/sdb       \
                --label=hdd.hdd1 /dev/sdc       \
                --label=hdd.hdd2 /dev/sdd       \
                --label=hdd.hdd3 /dev/sde       \
                --label=hdd.hdd4 /dev/sdf       \
                --foreground_target=ssd	        \
                --promote_target=ssd            \
                --background_target=hdd
\end{verbatim} \end{quote}

\subsection{Mounting}

To mount a multi device filesystem, there are two options. You can specify all
component devices, separated by hyphens, e.g. 
\begin{quote} \begin{verbatim}
mount -t bcachefs /dev/sda:/dev/sdb:/dev/sdc /mnt
\end{verbatim} \end{quote}
Or, use the mount.bcachefs tool to mount by filesystem UUID. Still todo: improve
the mount.bcachefs tool to support mounting by filesystem label.

No special handling is needed for recovering from unclean shutdown. Journal
replay happens automatically, and diagnostic messages in the dmesg log will
indicate whether recovery was from clean or unclean shutdown.

The \texttt{-o degraded} option will allow a filesystem to be mounted without
all the the devices, but will fail if data would be missing. The
\texttt{-o very\_degraded} can be used to attempt mounting when data would be
missing.

Also relevant is the \texttt{-o nochanges} option. It disallows any and all
writes to the underlying devices, pinning dirty data in memory as necessary if
for example journal replay was necessary - think of it as a "super read-only"
mode. It can be used for data recovery, and for testing version upgrades.

The \texttt{-o verbose} enables additional log output during the mount process.

\subsection{Fsck}

It is possible to run fsck either in userspace with the \texttt{bcachefs fsck}
subcommand (also available as \texttt{fsck.bcachefs}, or in the kernel while
mounting by specifying the \texttt{-o fsck} mount option. In either case the
exact same fsck implementation is being run, only the environment is different.
Running fsck in the kernel at mount time has the advantage of somewhat better
performance, while running in userspace has the ability to be stopped with
ctrl-c and can prompt the user for fixing errors. To fix errors while running
fsck in the kernel, use the \texttt{-o fix\_errors} option.

The \texttt{-n} option passed to fsck implies the \texttt{-o nochanges} option;
\texttt{bcachefs fsck -ny} can be used to test filesystem repair in dry-run
mode.

\subsection{Status of data}

The \texttt{bcachefs fs usage} may be used to display filesystem usage broken
out in various ways. Data usage is broken out by type: superblock, journal,
btree, data, cached data, and parity, and by which sets of devices extents are
replicated across. We also give per-device usage which includes fragmentation
due to partially used buckets.

\subsection{Journal}

The journal has a number of tunables that affect filesystem performance. Journal
commits are fairly expensive operations as they require issuing FLUSH and FUA
operations to the underlying devices. By default, we issue a journal flush one
second after a filesystem update has been done; this is controlled with the
\texttt{journal\_flush\_delay} option, which takes a parameter in milliseconds.

Filesystem sync and fsync operations issue journal flushes; this can be disabled
with the \texttt{journal\_flush\_disabled} option - the
\texttt{journal\_flush\_delay} option will still apply, and in the event of a
system crash we will never lose more than (by default) one second of work. This
option may be useful on a personal workstation or laptop, and perhaps less
appropriate on a server.

The journal reclaim thread runs in the background, kicking off btree node writes
and btree key cache flushes to free up space in the journal. Even in the absence
of space pressure it will run slowly in the background: this is controlled by
the \texttt{journal\_reclaim\_delay} parameter, with a default of 100
milliseconds.

The journal should be sized sufficiently that bursts of activity do not fill up
the journal too quickly; also, a larger journal mean that we can queue up larger
btree writes. The \texttt{bcachefs device resize-journal} can be used for
resizing the journal on disk on a particular device - it can be used on a
mounted or unmounted filesystem.

In the future, we should implement a method to see how much space is currently
utilized in the journal.

\subsection{Device management}

\subsubsection{Filesystem resize}

A filesystem can be resized on a particular device with the
\texttt{bcachefs device resize} subcommand. Currently only growing is supported,
not shrinking.

\subsubsection{Device add/removal}

The following subcommands exist for adding and removing devices from a mounted
filesystem:
\begin{itemize}
	\item \texttt{bcachefs device add}: Formats and adds a new device to an
		existing filesystem.
	\item \texttt{bcachefs device remove}: Permenantly removes a device from
		an existing filesystem.
	\item \texttt{bcachefs device online}: Connects a device to a running
		filesystem that was mounted without it (i.e. in degraded mode)
	\item \texttt{bcachefs device offline}: Disconnects a device from a
		mounted filesystem without removing it.
	\item \texttt{bcachefs device evacuate}: Migrates data off of a
		particular device to prepare for removal, setting it read-only
		if necessary.
	\item \texttt{bcachefs device set-state}: Changes the state of a member
		device: one of rw (readwrite), ro (readonly), failed, or spare.

		A failed device is considered to have 0 durability, and replicas
		on that device won't be counted towards the number of replicas
		an extent should have by rereplicate - however, bcachefs will
		still attempt to read from devices marked as failed.
\end{itemize}

The \texttt{bcachefs device remove}, \texttt{bcachefs device offline} and
\texttt{bcachefs device set-state} commands take force options for when they
would leave the filesystem degraded or with data missing. Todo: regularize and
improve those options.

\subsection{Data management}

\subsubsection{Data rereplicate}

The \texttt{bcachefs data rereplicate} command may be used to scan for extents
that have insufficient replicas and write additional replicas, e.g. after a
device has been removed from a filesystem or after replication has been enabled
or increased.

\subsubsection{Rebalance}

To be implemented: a command for moving data between devices to equalize usage
on each device. Not normally required because the allocator attempts to equalize
usage across devices as it stripes, but can be necessary in certain scenarios -
i.e. when a two-device filesystem with replication enabled that is very full has
a third device added.

\subsubsection{Scrub}

To be implemented: a command for reading all data within a filesystem and
ensuring that checksums are valid, fixing bitrot when a valid copy can be found.

\section{Options}

Most bcachefs options can be set filesystem wide, and a significant subset can
also be set on inodes (files and directories), overriding the global defaults.
Filesystem wide options may be set when formatting, when mounting, or at runtime
via \texttt{/sys/fs/bcachefs/<uuid>/options/}. When set at runtime via sysfs the
persistent options in the superblock are updated as well; when options are
passed as mount parameters the persistent options are unmodified.

\subsection{File and directory options}

Options set on inodes (files and directories) are automatically inherited by
their descendants, and inodes also record whether a given option was explicitly
set or inherited from their parent. When renaming a directory would cause
inherited attributes to change we fail the rename with -EXDEV, causing userspace
to do the rename file by file so that inherited attributes stay consistent.

Inode options are available as extended attributes. The options that have been
explicitly set are available under the \texttt{bcachefs} namespace, and the effective
options (explicitly set and inherited options) are available under the
\texttt{bcachefs\_effective} namespace. Examples of listing options with the
getfattr command:

\begin{quote} \begin{verbatim}
$ getfattr -d -m '^bcachefs\.' filename
$ getfattr -d -m '^bcachefs_effective\.' filename
\end{verbatim} \end{quote}

Options may be set via the extended attribute interface, but it is preferable to
use the \texttt{bcachefs setattr} command as it will correctly propagate options
recursively.

\subsection{Full option list}

\begin{tabbing}
\hspace{0.2in} \= \kill
	\texttt{block\_size}		\` \textbf{format}			\\
	\> \parbox{4.3in}{Filesystem block size (default 4k)}			\\ \\

	\texttt{btree\_node\_size}	\` \textbf{format}			\\
	\> Btree node size, default 256k					\\ \\

	\texttt{errors}			\` \textbf{format,mount,rutime}		\\
	\> Action to take on filesystem error					\\ \\

	\texttt{metadata\_replicas}	\` \textbf{format,mount,runtime}	\\
	\> Number of replicas for metadata (journal and btree)			\\ \\

	\texttt{data\_replicas}		\` \textbf{format,mount,runtime,inode}	\\
	\> Number of replicas for user data					\\ \\

	\texttt{replicas}		\` \textbf{format}			\\
	\> Alias for both metadata\_replicas and data\_replicas			\\ \\

	\texttt{metadata\_checksum}	\` \textbf{format,mount,runtime}	\\
	\> Checksum type for metadata writes					\\ \\

	\texttt{data\_checksum}		\` \textbf{format,mount,runtime,inode}	\\
	\> Checksum type for data writes					\\ \\

	\texttt{compression}		\` \textbf{format,mount,runtime,inode}	\\
	\> Compression type							\\ \\

	\texttt{background\_compression} \` \textbf{format,mount,runtime,inode}	\\
	\> Background compression type						\\ \\

	\texttt{str\_hash}		\` \textbf{format,mount,runtime,inode}	\\
	\> Hash function for string hash tables (directories and xattrs)	\\ \\

	\texttt{metadata\_target}	\` \textbf{format,mount,runtime,inode}	\\
	\> Preferred target for metadata writes					\\ \\

	\texttt{foreground\_target}	\` \textbf{format,mount,runtime,inode}	\\
	\> Preferred target for foreground writes				\\ \\

	\texttt{background\_target}	\` \textbf{format,mount,runtime,inode}	\\
	\> Target for data to be moved to in the background			\\ \\

	\texttt{promote\_target}	\` \textbf{format,mount,runtime,inode}	\\
	\> Target for data to be copied to on read				\\ \\

	\texttt{erasure\_code}		\` \textbf{format,mount,runtime,inode}	\\
	\> Enable erasure coding						\\ \\

	\texttt{inodes\_32bit}		\` \textbf{format,mount,runtime}	\\
	\> Restrict new inode numbers to 32 bits				\\ \\

	\texttt{shard\_inode\_numbers}	\` \textbf{format,mount,runtime}	\\
	\> Use CPU id for high bits of new inode numbers. 			\\ \\

	\texttt{wide\_macs}		\` \textbf{format,mount,runtime}	\\
	\> Store full 128 bit cryptographic MACs (default 80)			\\ \\

	\texttt{inline\_data}		\` \textbf{format,mount,runtime}	\\
	\> Enable inline data extents (default on)				\\ \\

	\texttt{journal\_flush\_delay}	\` \textbf{format,mount,runtime}	\\
	\> Delay in milliseconds before automatic journal commit (default 1000)	\\ \\

	\texttt{journal\_flush\_disabled}\`\textbf{format,mount,runtime}	\\
	\> \begin{minipage}{4.3in}Disables journal flush on sync/fsync.
		\texttt{journal\_flush\_delay}	remains in effect, thus with the
		default setting not more than 1 second of work will be lost.
	\end{minipage}								\\ \\

	\texttt{journal\_reclaim\_delay}\` \textbf{format,mount,runtime}	\\
	\> Delay in milliseconds before automatic journal reclaim		\\ \\

	\texttt{acl}			\` \textbf{format,mount}		\\
	\> Enable POSIX ACLs							\\ \\

	\texttt{usrquota}		\` \textbf{format,mount}		\\
	\> Enable user quotas							\\ \\

	\texttt{grpquota}		\` \textbf{format,mount}		\\
	\> Enable group quotas							\\ \\

	\texttt{prjquota}		\` \textbf{format,mount}		\\
	\> Enable project quotas						\\ \\

	\texttt{degraded}		\` \textbf{mount}			\\
	\> Allow mounting with data degraded					\\ \\

	\texttt{very\_degraded}		\` \textbf{mount}			\\
	\> Allow mounting with data missing					\\ \\

	\texttt{verbose}		\` \textbf{mount}			\\
	\> Extra debugging info during mount/recovery				\\ \\

	\texttt{fsck}			\` \textbf{mount}			\\
	\> Run fsck during mount						\\ \\

	\texttt{fix\_errors}		\` \textbf{mount}			\\
	\> Fix errors without asking during fsck				\\ \\

	\texttt{ratelimit\_errors}	\` \textbf{mount}			\\
	\> Ratelimit error messages during fsck					\\ \\

	\texttt{read\_only}		\` \textbf{mount}			\\
	\> Mount in read only mode						\\ \\

	\texttt{nochanges}		\` \textbf{mount}			\\
	\> Issue no writes, even for journal replay				\\ \\

	\texttt{norecovery}		\` \textbf{mount}			\\
	\> Don't replay the journal (not recommended)				\\ \\

	\texttt{noexcl}			\` \textbf{mount}			\\
	\> Don't open devices in exclusive mode					\\ \\

	\texttt{version\_upgrade}	\` \textbf{mount}			\\
	\> Upgrade on disk format to latest version				\\ \\

	\texttt{discard}		\` \textbf{device}			\\
	\> Enable discard/TRIM support						\\ \\
\end{tabbing}

\subsection{Error actions}
The \texttt{errors} option is used for inconsistencies that indicate some sort
of a bug. Valid error actions are:
\begin{description}
	\item[{\tt continue}] Log the error but continue normal operation
	\item[{\tt ro}] Emergency read only, immediately halting any changes
		to the filesystem on disk
	\item[{\tt panic}] Immediately halt the entire machine, printing a
		backtrace on the system console
\end{description}

\subsection{Checksum types}
Valid checksum types are:
\begin{description}
	\item[{\tt none}]
	\item[{\tt crc32c}] (default)
	\item[{\tt crc64}]
\end{description}

\subsection{Compression types}
Valid compression types are:
\begin{description}
	\item[{\tt none}] (default)
	\item[{\tt lz4}]
	\item[{\tt gzip}]
	\item[{\tt zstd}]
\end{description}

\subsection{String hash types}
Valid hash types for string hash tables are:
\begin{description}
	\item[{\tt crc32c}]
	\item[{\tt crc64}]
	\item[{\tt siphash}] (default)
\end{description}

\section{Debugging tools}

\subsection{Sysfs interface}

Mounted filesystems are available in sysfs at \texttt{/sys/fs/bcachefs/<uuid>/}
with various options, performance counters and internal debugging aids.

\subsubsection{Options}

Filesystem options may be viewed and changed via \\
\texttt{/sys/fs/bcachefs/<uuid>/options/}, and settings changed via sysfs will
be persistently changed in the superblock as well.

\subsubsection{Time stats}

bcachefs tracks the latency and frequency of various operations and events, with
quantiles for latency/duration in the
\texttt{/sys/fs/bcachefs/<uuid>/time\_stats/} directory.

\begin{description}
	\item \texttt{blocked\_allocate} \\
		Tracks when allocating a bucket must wait because none are
		immediately available, meaning the copygc thread is not keeping
		up with evacuating mostly empty buckets or the allocator thread
		is not keeping up with invalidating and discarding buckets.

	\item \texttt{blocked\_allocate\_open\_bucket} \\
		Tracks when allocating a bucket must wait because all of our
		handles for pinning open buckets are in use (we statically
		allocate 1024).

	\item \texttt{blocked\_journal} \\
		Tracks when getting a journal reservation must wait, either
		because journal reclaim isn't keeping up with reclaiming space
		in the journal, or because journal writes are taking too long to
		complete and we already have too many in flight.

	\item \texttt{btree\_gc} \\
		Tracks when the btree\_gc code must walk the btree at runtime -
		for recalculating the oldest outstanding generation number of
		every bucket in the btree.

	\item \texttt{btree\_lock\_contended\_read}
	\item \texttt{btree\_lock\_contended\_intent}
	\item \texttt{btree\_lock\_contended\_write} \\
		Track when taking a read, intent or write lock on a btree node
		must block.

	\item \texttt{btree\_node\_mem\_alloc} \\
		Tracks the total time to allocate memory in the btree node cache
		for a new btree node.

	\item \texttt{btree\_node\_split} \\
		Tracks btree node splits - when a btree node becomes full and is
		split into two new nodes

	\item \texttt{btree\_node\_compact} \\
		Tracks btree node compactions - when a btree node becomes full
		and needs to be compacted on disk.

	\item \texttt{btree\_node\_merge} \\
		Tracks when two adjacent btree nodes are merged.

	\item \texttt{btree\_node\_sort} \\
		Tracks sorting and resorting entire btree nodes in memory,
		either after reading them in from disk or for compacting prior
		to creating a new sorted array of keys.

	\item \texttt{btree\_node\_read} \\
		Tracks reading in btree nodes from disk.

	\item \texttt{btree\_interior\_update\_foreground} \\
		Tracks foreground time for btree updates that change btree
		topology - i.e. btree node splits, compactions and merges; the
		duration measured roughly corresponds to lock held time.

	\item \texttt{btree\_interior\_update\_total} \\
		Tracks time to completion for topology changing btree updates;
		first they have a foreground part that updates btree nodes in
		memory, then after the new nodes are written there is a
		transaction phase that records an update to an interior node or
		a new btree root as well as changes to the alloc btree.

	\item \texttt{data\_read} \\
		Tracks the core read path - looking up a request in the extents
		(and possibly also reflink) btree, allocating bounce buffers if
		necessary, issuing reads, checksumming, decompressing, decrypting,
		and delivering completions.

	\item \texttt{data\_write} \\
		Tracks the core write path - allocating space on disk for a new
		write, allocating bounce buffers if necessary,
		compressing, encrypting, checksumming, issuing writes, and
		updating the extents btree to point to the new data.

	\item \texttt{data\_promote} \\
		Tracks promote operations, which happen when a read operation
		writes an additional cached copy of an extent to
		\texttt{promote\_target}. This is done asynchronously from the
		original read.

	\item \texttt{journal\_flush\_write} \\
		Tracks writing of flush journal entries to disk, which first
		issue cache flush operations to the underlying devices then
		issue the journal writes as FUA writes. Time is tracked starting
		from after all journal reservations have released their
		references or the completion of the previous journal write.

	\item \texttt{journal\_noflush\_write} \\
		Tracks writing of non-flush journal entries to disk, which do
		not issue cache flushes or FUA writes.

	\item \texttt{journal\_flush\_seq} \\
		Tracks time to flush a journal sequence number to disk by
		filesystem sync and fsync operations, as well as the allocator
		prior to reusing buckets when none that do not need flushing are
		available.
\end{description}

\subsubsection{Internals}

\begin{description}
	\item \texttt{btree\_cache} \\
		Shows information on the btree node cache: number of cached
		nodes, number of dirty nodes, and whether the cannibalize lock
		(for reclaiming cached nodes to allocate new nodes) is held.

	\item \texttt{dirty\_btree\_nodes} \\
		Prints information related to the interior btree node update
		machinery, which is responsible for ensuring dependent btree
		node writes are ordered correctly.

		For each dirty btree node, prints:
		\begin{itemize}
			\item Whether the \texttt{need\_write} flag is set
			\item The level of the btree node
			\item The number of sectors written
			\item Whether writing this node is blocked, waiting for
				other nodes to be written
			\item Whether it is waiting on a btree\_update to
				complete and make it reachable on-disk
		\end{itemize}

	\item \texttt{btree\_key\_cache} \\
		Prints infromation on the btree key cache: number of freed keys
		(which must wait for a sRCU barrier to complete before being
		freed), number of cached keys, and number of dirty keys.

	\item \texttt{btree\_transactions} \\
		Lists each running btree transactions that has locks held,
		listing which nodes they have locked and what type of lock, what
		node (if any) the process is blocked attempting to lock, and
		where the btree transaction was invoked from.

	\item \texttt{btree\_updates} \\
		Lists outstanding interior btree updates: the mode (nothing
		updated yet, or updated a btree node, or wrote a new btree root,
		or was reparented by another btree update), whether its new
		btree nodes have finished writing, its embedded closure's
		refcount (while nonzero, the btree update is still waiting), and
		the pinned journal sequence number.

	\item \texttt{journal\_debug} \\
		Prints a variety of internal journal state.

	\item \texttt{journal\_pins}
		Lists items pinning journal entries, preventing them from being
		reclaimed.

	\item \texttt{new\_stripes} \\
		Lists new erasure-coded stripes being created.

	\item \texttt{stripes\_heap} \\
		Lists erasure-coded stripes that are available to be reused.

	\item \texttt{open\_buckets} \\
		Lists buckets currently being written to, along with data type
		and refcount.

	\item \texttt{io\_timers\_read} \\
	\item \texttt{io\_timers\_write} \\
		Lists outstanding IO timers - timers that wait on total reads or
		writes to the filesystem.

	\item \texttt{trigger\_journal\_flush} \\
		Echoing to this file triggers a journal commit.

	\item \texttt{trigger\_gc} \\
		Echoing to this file causes the GC code to recalculate each
		bucket's oldest\_gen field.

	\item \texttt{prune\_cache} \\
		Echoing to this file prunes the btree node cache.

	\item \texttt{read\_realloc\_races} \\
		This counts events where the read path reads an extent and
		discovers the bucket that was read from has been reused while
		the IO was in flight, causing the read to be retried.

	\item \texttt{extent\_migrate\_done} \\
		This counts extents moved by the core move path, used by copygc
		and rebalance.

	\item \texttt{extent\_migrate\_raced} \\
		This counts extents that the move path attempted to move but no
		longer existed when doing the final btree update.
\end{description}

\subsubsection{Unit and performance tests}

Echoing into \texttt{/sys/fs/bcachefs/<uuid>/perf\_test} runs various low level
btree tests, some intended as unit tests and others as performance tests. The
syntax is
\begin{quote} \begin{verbatim}
	echo <test_name> <nr_iterations> <nr_threads> > perf_test
\end{verbatim} \end{quote}

When complete, the elapsed time will be printed in the dmesg log. The full list
of tests that can be run can be found near the bottom of
\texttt{fs/bcachefs/tests.c}.

\subsection{Debugfs interface}

The contents of every btree, as well as various internal per-btree-node
information, are available under \texttt{/sys/kernel/debug/bcachefs/<uuid>/}.

For every btree, we have the following files:

\begin{description}
	\item \textit{btree\_name} \\
		Entire btree contents, one key per line

	\item \textit{btree\_name}\texttt{-formats} \\
		Information about each btree node: the size of the packed bkey
		format, how full each btree node is, number of packed and
		unpacked keys, and number of nodes and failed nodes in the
		in-memory search trees.

	\item \textit{btree\_name}\texttt{-bfloat-failed} \\
		For each sorted set of keys in a btree node, we construct a
		binary search tree in eytzinger layout with compressed keys.
		Sometimes we aren't able to construct a correct compressed
		search key, which results in slower lookups; this file lists the
		keys that resulted in these failed nodes.
\end{description}

\subsection{Listing and dumping filesystem metadata}

\subsubsection{bcachefs show-super}

This subcommand is used for examining and printing bcachefs superblocks. It
takes two optional parameters:
\begin{description}
	\item \texttt{-l}: Print superblock layout, which records the amount of
		space reserved for the superblock and the locations of the
		backup superblocks.
	\item \texttt{-f, --fields=(fields)}: List of superblock sections to
		print, \texttt{all} to print all sections.
\end{description}

\subsubsection{bcachefs list}

This subcommand gives access to the same functionality as the debugfs interface,
listing btree nodes and contents, but for offline filesystems.

\subsubsection{bcachefs list\_journal}

This subcommand lists the contents of the journal, which primarily records btree
updates ordered by when they occured.

\subsubsection{bcachefs dump}

This subcommand can dump all metadata in a filesystem (including multi device
filesystems) as qcow2 images: when encountering issues that \texttt{fsck} can
not recover from and need attention from the developers, this makes it possible
to send the developers only the required metadata. Encrypted filesystems must
first be unlocked with \texttt{bcachefs remove-passphrase}.

\section{ioctl interface}

This section documents bcachefs-specific ioctls:

\begin{description}
	\item \texttt{BCH\_IOCTL\_QUERY\_UUID} \\
		Returs the UUID of the filesystem: used to find the sysfs
		directory given a path to a mounted filesystem.

	\item \texttt{BCH\_IOCTL\_FS\_USAGE} \\
		Queries filesystem usage, returning global counters and a list
		of counters by \texttt{bch\_replicas} entry.

	\item \texttt{BCH\_IOCTL\_DEV\_USAGE} \\
		Queries usage for a particular device, as bucket and sector
		counts broken out by data type.

	\item \texttt{BCH\_IOCTL\_READ\_SUPER} \\
		Returns the filesystem superblock, and optionally the superblock
		for a particular device given that device's index.

	\item \texttt{BCH\_IOCTL\_DISK\_ADD} \\
		Given a path to a device, adds it to a mounted and running
		filesystem. The device must already have a bcachefs superblock;
		options and parameters are read from the new device's superblock
		and added to the member info section of the existing
		filesystem's superblock.

	\item \texttt{BCH\_IOCTL\_DISK\_REMOVE} \\
		Given a path to a device or a device index, attempts to remove
		it from a mounted and running filesystem. This operation
		requires walking the btree to remove all references to this
		device, and may fail if data would become degraded or lost,
		unless appropriate force flags are set.

	\item \texttt{BCH\_IOCTL\_DISK\_ONLINE} \\
		Given a path to a device that is a member of a running
		filesystem (in degraded mode), brings it back online.

	\item \texttt{BCH\_IOCTL\_DISK\_OFFLINE} \\
		Given a path or device index of a device in a multi device
		filesystem, attempts to close it without removing it, so that
		the device may be re-added later and the contents will still be
		available.

	\item \texttt{BCH\_IOCTL\_DISK\_SET\_STATE} \\
		Given a path or device index of a device in a multi device
		filesystem, attempts to set its state to one of read-write,
		read-only, failed or spare. Takes flags to force if the
		filesystem would become degraded.

	\item \texttt{BCH\_IOCTL\_DISK\_GET\_IDX} \\
	\item \texttt{BCH\_IOCTL\_DISK\_RESIZE} \\
	\item \texttt{BCH\_IOCTL\_DISK\_RESIZE\_JOURNAL} \\
	\item \texttt{BCH\_IOCTL\_DATA} \\
		Starts a data job, which walks all data and/or metadata in a
		filesystem performing, performaing some operation on each btree
		node and extent. Returns a file descriptor which can be read
		from to get the current status of the job, and closing the file
		descriptor (i.e. on process exit stops the data job.

	\item \texttt{BCH\_IOCTL\_SUBVOLUME\_CREATE} \\
	\item \texttt{BCH\_IOCTL\_SUBVOLUME\_DESTROY} \\
	\item \texttt{BCHFS\_IOC\_REINHERIT\_ATTRS} \\
\end{description}

\section{On disk format}

\subsection{Superblock}

The superblock is the first thing to be read when accessing a bcachefs
filesystem. It is located 4kb from the start of the device, with redundant
copies elsewhere - typically one immediately after the first superblock, and one
at the end of the device.

The \texttt{bch\_sb\_layout} records the amount of space reserved for the
superblock as well as the locations of all the superblocks. It is included with
every superblock, and additionally written 3584 bytes from the start of the
device (512 bytes before the first superblock).

Most of the superblock is identical across each device. The exceptions are the
\texttt{dev\_idx} field, and the journal section which gives the location of the
journal.

The main section of the superblock contains UUIDs, version numbers, number of
devices within the filesystem and device index, block size, filesystem creation
time, and various options and settings. The superblock also has a number of
variable length sections:

\begin{description}
	\item \texttt{BCH\_SB\_FIELD\_journal} \\
		List of buckets used for the journal on this device.

	\item \texttt{BCH\_SB\_FIELD\_members} \\
		List of member devices, as well as per-device options and
		settings, including bucket size, number of buckets and time when
		last mounted.

	\item \texttt{BCH\_SB\_FIELD\_crypt} \\
		Contains the main chacha20 encryption key, encrypted by the
		user's passphrase, as well as key derivation function settings.

	\item \texttt{BCH\_SB\_FIELD\_replicas} \\
		Contains a list of replica entries, which are lists of devices
		that have extents replicated across them. 

	\item \texttt{BCH\_SB\_FIELD\_quota} \\
		Contains timelimit and warnlimit fields for each quota type
		(user, group and project) and counter (space, inodes).

	\item \texttt{BCH\_SB\_FIELD\_disk\_groups} \\
		Formerly referred to as disk groups (and still is throughout the
		code); this section contains device label strings and records
		the tree structure of label paths, allowing a label once parsed
		to be referred to by integer ID by the target options.

	\item \texttt{BCH\_SB\_FIELD\_clean} \\
		When the filesystem is clean, this section contains a list of
		journal entries that are normally written with each journal
		write (\texttt{struct jset}): btree roots, as well as filesystem
		usage and read/write counters (total amount of data read/written
		to this filesystem). This allows reading the journal to be
		skipped after clean shutdowns.
\end{description}

\subsection{Journal}

Every journal write (\texttt{struct jset}) contains a list of entries:
\texttt{struct jset\_entry}. Below are listed the various journal entry types.

\begin{description}
	\item \texttt{BCH\_JSET\_ENTRY\_btree\_key} \\
		This entry type is used to record every btree update that
		happens. It contains one or more btree keys (\texttt{struct
		bkey}), and the \texttt{btree\_id} and \texttt{level} fields of
		\texttt{jset\_entry} record the btree ID and level the key
		belongs to.

	\item \texttt{BCH\_JSET\_ENTRY\_btree\_root} \\
		This entry type is used for pointers btree roots. In the current
		implementation, every journal write still records every btree
		root, although that is subject to change. A btree root is a bkey
		of type \texttt{KEY\_TYPE\_btree\_ptr\_v2}, and the btree\_id
		and level fields of \texttt{jset\_entry} record the btree ID and
		depth.

	\item \texttt{BCH\_JSET\_ENTRY\_clock} \\
		Records IO time, not wall clock time - i.e. the amount of reads
		and writes, in 512 byte sectors since the filesystem was
		created.

	\item \texttt{BCH\_JSET\_ENTRY\_usage} \\
		Used for certain persistent counters: number of inodes, current
		maximum key version, and sectors of persistent reservations.

	\item \texttt{BCH\_JSET\_ENTRY\_data\_usage} \\
		Stores replica entries with a usage counter, in sectors.

	\item \texttt{BCH\_JSET\_ENTRY\_dev\_usage} \\
		Stores usage counters for each device: sectors used and buckets
		used, broken out by each data type.
\end{description}

\subsection{Btrees}

\subsection{Btree keys}

\begin{description}
	\item \texttt{KEY\_TYPE\_deleted}
	\item \texttt{KEY\_TYPE\_whiteout}
	\item \texttt{KEY\_TYPE\_error}
	\item \texttt{KEY\_TYPE\_cookie}
	\item \texttt{KEY\_TYPE\_hash\_whiteout}
	\item \texttt{KEY\_TYPE\_btree\_ptr}
	\item \texttt{KEY\_TYPE\_extent}
	\item \texttt{KEY\_TYPE\_reservation}
	\item \texttt{KEY\_TYPE\_inode}
	\item \texttt{KEY\_TYPE\_inode\_generation}
	\item \texttt{KEY\_TYPE\_dirent}
	\item \texttt{KEY\_TYPE\_xattr}
	\item \texttt{KEY\_TYPE\_alloc}
	\item \texttt{KEY\_TYPE\_quota}
	\item \texttt{KEY\_TYPE\_stripe}
	\item \texttt{KEY\_TYPE\_reflink\_p}
	\item \texttt{KEY\_TYPE\_reflink\_v}
	\item \texttt{KEY\_TYPE\_inline\_data}
	\item \texttt{KEY\_TYPE\_btree\_ptr\_v2}
	\item \texttt{KEY\_TYPE\_indirect\_inline\_data}
	\item \texttt{KEY\_TYPE\_alloc\_v2}
	\item \texttt{KEY\_TYPE\_subvolume}
	\item \texttt{KEY\_TYPE\_snapshot}
	\item \texttt{KEY\_TYPE\_inode\_v2}
	\item \texttt{KEY\_TYPE\_alloc\_v3}
\end{description}

\end{document}
